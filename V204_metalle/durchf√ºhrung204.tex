\section{Durchführung}

Der Versuchsaufbau besteht aus einem Netzgerät,
einem Datenlogger, einer Platine mit vier verschiedenen darauf montierten Metallplatten und
einem Peltier-Element in der Mitte und einem Drucker.
Das wichtigste Stück des Versuchsaufbaus ist das Peltier-Element.\\
Ähnlich eines galvanischen Elements sind auch hier zweitelige
Schaltelemente aneinandergereiht.
In diesen einzelnen Elementen liegen zwei Halbleiter in Kontakt aneinander, die energetisch unterschiedlich gelegene Leitungsbändern besitzen.
Abhängig von der Stromrichtung die durch diese Elemente fließt muss also, 
wenn ein Elektron den Übergang in den Halbleiter mit dem höheren Leitungsband schaffen muss Energie aufgenommen werden,
was durch Reduzierung der Umgebungswärme geschieht.\\

Als erstes werden die Abstände der einzelnen Thermoelemente
als $\num{3}\si{\centi\metre}$ festgestellt, das Netzgerät
angeschlossen und die Messapparatur eingeschaltet und
über das Temperatur-Array mit dem Aufbau verbunden. 
Der Heiß-/Kalt-Schalter wird aus "COOL" gestellt.\\
Vor Beginn der ersten Messung muss die Abtastrate noch auf $\num{5}\si{\second}$ gestellt werden, dann kann die Isolierung auf die Platten gelegt und die Spannung auf $\num{8}\si{\Volt}$
gelegt werden. Zum Beginn der Mesung muss nun noch der Schalter auf "HEAT" gestellt werden und
das Messgerät muss seine Messung starten.\\
Nach $\num{700}\si{\second}$ wird dann an $T_1$,
$T_4$, $T_5$, $T_8$ die Temperatur genommen.
Wenn $T_7$ etwa $\num{45}\si{\degreeCelsius}$
oder kaum noch Wärme zugeführt wird, wird die Messung gestoppt.
Nun können die Isolierungen entfernt werden,
sodass die Platten abkühlen können und
währenddessen können Plots für $T_1$ und $T_4$,
sowie für $T_5$ und $T_8$ erstellt werden.
Außerdemm werden noch die Drucke für die Differenzen $T_2-T_1$
und $T_8-T_7$ benötigt.\\
\\
Zur Duchführung der dynamischen Methode muss nun die
Abtastrate auf $\num{2}\si{\second}$ und
die Spannunng auf $\num{10,5}\si{\Volt}$ erhöht werden.
Auch die Isolierungen werden wieder aufgestezt.\\
Nun wird jeweils $\num{40}\si{\second}$ lang erhitzt und dann wieder die selbe Zeit gekühlt. Dieser Vorgang wird zehn mal wiederholt.
Hier soll nur das Verhalten des breiten Messingstabes ($T_1$ und $T_2$) gedruckt werden.
Dieses Vorgehen wird mit fünfacher Periodendauer
($\num{400}\si{\second}$) und für den Edelstahlstab wiederholt.
Hierbei sollte keins der Thermoelemente $\num{80}\si{\degreeCelsius}$ überschreiten.


