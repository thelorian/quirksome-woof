\section{Zielsetzung}
In diesem Versuch wir untersucht werden wie verschiedene
Metalle (Messing,Aluminium und Edelstahl) auf ein statisches und ein sich veränderndes Temperaturgefälle reagieren,
insbesonndere welche Metalle besonders gute Wärmeleitfähigkeit besitzen. 
Außerdem wird anhand zweier verschiedener Messingstäbe untersucht, inwiefen 
sich die Form der Metallteile auf ihre Wärmeleitfähigkeit auswirkt.


\section{Theorie}
\label{sec:Theorie}

Wenn in einem System ein Temperaturgefälle besteht, tritt entlang diesen Gefälles ein Wärmetransport auf.
Dieser Austausch kann auf verschiedene Weisen einteten: durch Konvektion, durch Wärmstrahlung, und durch Wärmeleitung.
In diesem Experiment wird sich auf die Wärmeleitung beschränkt. Diese kann in Festkörpern durch Anregung der Gittersturktur (Phononen) und über freie Elektronen geschehen.
Wegen der hohen dicht von freien Elektronen in Metallen kann die Phononenübertragung vernachlässigt werden.

Bei einem Stab mit Länge $L$ und Querschnitt $A$ aus einem Material mit der Dichte $\rho$
und spezifischer Wärme $c$ bei dem da Gefälle entlang des Stabes besteht, ist die in $\dif{t}$ durch den Querschnitt des Stabes
fließt
\begin{equation}
  \dif{Q} = -\kappa A \frac{\del{T}}{\del{x}}\dif{t}
\end{equation}

Der Proportionalitätsfaktor $\kappa$ ist dabei Wärmeleitfähigkeit. Für die Wärmestromdichte folgt dann
\begin{equation}
  j_w = -\kappa \frac{\del{T}}{\del{x}}.
\end{equation}

Zusammen mit der Kontinuitätsgleichung mit der Wärmedichte $\rho_Q = \frac{\del{Q}}{\del{V}}$
\begin{equation}
  \frac{\del{\rho_Q}}{\del{V}}+\nabla \vec{j_w} = 0
\end{equation}
folgt dann 
\begin{equation}
  \frac{\del{T}}{\del{t}} = \frac{\kappa}{\rho\cdot c}\frac{\del^2{T}}{\del{x}^2}
\end{equation}
die Wärmeleitungsgleichung, die den Zusammenhang zwischen räumlicher und zeitlicher Entwicklung der Temperaturvertilung herstellt.
Der Proportionalitätsfaktor wird als Temperaturleitfähigkeit bezeichnet und gibt die Stärke des Wärmestroms an. 
Wird wie im Experiment ein langer Stab verwendet und periodisch erwärmt und gekühlt ergibt sich die Lösung dieser DGL eine sich räumlich und zeitlich fortpflanzende Temperaturwelle:
\begin{equation}
  T(x,t) = T_{\text{max}}exp(-\sqrt{\frac{\omega\rho c}{2\kappa}}x) \cdot \cos{\omega t -\sqrt{\frac{\omega\rho c}{2\kappa}}}.
\end{equation}
Nach der Dispersionsrealtion für fortschreitende Wellen ergibt sich die Ausbreitungs- bzw. Phasengeschwindigkeit
\begin{equation}
  v = \frac{\omega}{k} = \frac{\omega}{\sqrt{\frac{\omega\rho c}{2\kappa}}} = \sqrt{\frac{2\kappa\omega}{\rho c}}
\end{equation}

Die Dämpfung der Welle ergibt sich als Differenz der Amplitude and einem Punkt nah am Erzeugungspunkt und einem weiteren, weiter entfernten:
\begin{align}
  A_{\text{nah}} &= T(x_{\text{nah}},t) \\
  A_{\text{fern}} &= T(x_{\text{fern}},t)\\
  \text{und mit den bekannten Identitäten bei periodischem Verhalten($T^*$ ist die Periodendauer, $\varphi$ die Phase)}
  \omega &= 2\pi T^*
  \varphi&= 2\pi \frac{\Delta t}{T^*}
  \text{ergibt sich die Formel für die Wärmeleitfähigkeit}
  \kappa &= \frac{\rho c (\Delta x)^2}{2\Delta t\cdot \ln{\frac{A_{\text{nah}}}{A_{\text{fern}}}}}
\end{align}








