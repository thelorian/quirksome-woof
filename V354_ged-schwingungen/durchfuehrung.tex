% !TEX program = lualatex
\documentclass{scrartcl}
\usepackage{polyglossia}
\begin{document}


    \section{Durchführung}
    \label{sec:Durchführung}
    Als erstes wird an den gegebenen Schwingkreis mit vorab bekannten Größen L,R,C
    eine Nadelpulsspannung angelegt und parallel zum Kondensator wird ein Oszilloskop angeschlossen.
    Nun stellt man die Frequenz des Nadelpuls' so ein, dass man nach dem jeweiligen Puls Relaxionsverhalten beobachten kann
    (die Spannung sollte mindestens um $\frac{1}{e}$ abgenommen haben).
    Von der enstehenden Abbildung einer gedäpften (Co-)Sinusschwingung wird der Thermodruck genommen,
    und bei den Extrema der Funktion werden jeweils Zeit seit dem Nadelpuls und Amplitude gemessen.

    Um den Grenzwiderstand $R_ap$ für den aperiodischen Grenzfall zu bestimmen wird nun,
    anstatt einem statischen R ein regelbarer Widerstand in den Schwingkreis geschaltet.
    Dieser wird von seinem Maximalwert so weit reduziert, dass sich eine sichtbare Überschwingung ausbildet
    und dann zurück auf eben den Wert eingestellt wo das Oszilloskop das erste mal eindeutig dem relaxionsverhalten eines geladenen Kondensators gleicht
    Der an diesem Punkt abzulesende Widerstand wird notiert.

    Zur Messung des Verhaltens des getriebenen Schwingkreises wird eine Wechselspannung and den Schwingkreis gelegt
    und zum Abgleich auf den zweiten Kanal des Oszilloskops gespeist.
    Als erstes wird die Amplitude der angelegten Wechselspannung aufgenommen
    und im Folgenden wird die Frequenz $\nu$ am Sinusgenerator variiert.
    Es sollte darauf geachtet werden, dass die Messabstände bei stärkeren Änderungen entsprechend kleiner zu wählen sind.

    Als Abschluss wird $\Phi(\omega)$ bestimmt die Phasenverschiebung, die der Schwingkreis aufgrund seiner Impedanz hervorruft.
    Zunächst wird die Erregerfrequenz klein gehalten, sodass die Erregung überwiegt und beide Wellen phasengleich sind
    und justiert das Oszilloskop so, dass bei beiden Wellen die Nulldurchgänge im selben punkt liegen.
    Und nun variiert man wie zuvor die Frequenz und misst die entstehenden Abstände zwischen den Nulldurchgängen.


\end{document}
