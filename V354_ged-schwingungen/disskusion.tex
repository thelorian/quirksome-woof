\documentclass[parskip=half, titlepage=firstiscover, captions=tableheading, bibliography=totoc]{scrartcl}
%optionen immer variieren tabelheading bei tabellen
%ohne parskip nur einrücken
%draft macht hüllen von bilder selbst wenn sie nicht existieren um einfach zeit bei compilieren zu sparen
%\usepackage{scrhack} % nach \documentclass
\usepackage{float}
\floatplacement{figure}{htbp}
\floatplacement{table}{htbp}
\usepackage[aux]{rerunfilecheck}
\usepackage{polyglossia}
\usepackage{biblatex}
\addbibresource{Tag3.bib}%name der bib-Datei hier einfügen !
\setmainlanguage{german}
\usepackage{longtable}
\usepackage{amsmath}
\usepackage{amssymb}
\usepackage{mathtools}
\usepackage{fontspec}
\usepackage[
math-style=ISO,
bold-style=ISO,
sans-style=italic,
nabla=upright,
partial=upright,
]{unicode-math}

\setmathfont{Latin Modern Math}
\usepackage{graphicx}
\usepackage{grffile}
\usepackage[font = scriptsize, labelfont = bf,margin={10pt,10pt}]{subcaption}
\usepackage[font = scriptsize, labelfont = bf,margin={10pt,10pt}]{caption}
%anstatt margin geht auch width = 10cm
\usepackage{mleftright}
%schöneres mit  \mleft (\mright)

\setlength{\delimitershortfall}{-1sp}
%bei vielen Klammern werden sie nun größer
\usepackage[locale=DE,separate-uncertainty=true,per-mode=symbol-or-fraction]{siunitx}
\usepackage{booktabs}
\usepackage{xfrac}
\usepackage{pdflscape}
%-> dazu wäre begin{landscape} etc nötig

%nur ein beispiel zu dieser änderung wenn man will
%gleiche sachen werden bei mathe oder text anders benutzt
%\let\vaccent=\v % alten Befehl kopieren
%\RenewDocumentCommand \v {} % Befehl überschreiben
%{
%\TextOrMath{
%\vaccent % Textmodus
%}{
%\symbf % Mathemodus
%}
%}

%\NewDocumentCommand \OverfullCenter {+m} {
%\noindent\makebox[\linewidth]{#1} }
%bei zu breiten pics zentrierte ausrichtung
%zu befehl wäre dann: \OverfullCenter{\includegraphics[width=\textwidth+15pt]{figures/
%Panorama.jpg}}

\AtBeginDocument{ % wird bei \begin{document} ausgeführt
\let\symIm=\Im % werden sonst wieder von unicode-math überschrieben
\RenewDocumentCommand \Re {}
{
\operatorname{Re}
}
\let\symIm=\Im
\RenewDocumentCommand \Im {}
{
\operatorname{Im}
}
}


\usepackage{fontspec}%nach amssymb
\usepackage[unicode]{hyperref}
\usepackage[shortcuts]{extdash} % nach hyperref, bookmark am Ende!
\usepackage{bookmark}

\usepackage[locale=DE,separate-uncertainty=true,per-mode=symbol-or-fraction]{siunitx}
\begin{document}
  \section{Disskusion}
  \label{sec:Disskusion}
Zum Abschluss werden nun Auffälligkeiten und Unterschiede von theoretischen und experimentellen Werten näher untersucht.
Der Unterschied vom erechneten effektiven Widerstand $R_{eff}$ und dem im Schaltkreis eingesetzten beträgt, wenn man kleine Messungenauigkeiten vernachlässigt
etwa $\SI{150}{\ohm}$, was der Wert der des Innenwiderstandes des Funktionengenerators mit $\SI{50}{\ohm}$ nicht vollständig erklärt werden kann.
Der große Unterschied könnte seinen Urpsrung in der mit $\SI{2000}{\hertz}$ vergleichsweise niedrigen angelegten Frequenz haben oder möglicherweise in einem ungenau gewählten Nullpunkt des Oszilloskops.

Der nächste zu betrachtende Wert ist $R_{ap}$ der Widerstand des aperiodischen Grenzfalls, hier ist ein Unterschied von etwa $\SI{1,4}{\ohm}$ zu vermerken, dieser rührt daher,
dass in der Rechnung der Widerstand anderer Bauteile vernachlässigt werden (z.B. Spule) und von der experimentellen Seite ein genaues Einstellen und Ablesen nicht sicher gegeben werden konnte.

Die folgende Werte zur Güte $q = 3,72 $ liegt mit Abweichung $\frac{\Delta q}{q} \approx 5\%$ im erwarteten Toleranzbereich.
Weiter liegt auch die Breite der Resonazkurve beim Untersuchen der Frequnezabhängigkeit der Kondensatorspannung mit $\frac{\Delta (\nu_+-\nu_-)}{\nu_+-\nu_-} \approx 10\%$ noch am Rand des zu erwartenden Bereichs;
mit der gegeben Schärfe des Oszilloskops, was für die sichtbar großen Fehler in (Abb. 2) verantwortlich ist sind beide Werte mit annehmbarer Genauigkeit bestimmt
Bei der Phasenuntersuchungen zeigen sich Fehler, die mit $0,05 Hz$ für \nu_{\text{res}}, $0,28 Hz$ für \nu_1 und $0,20 Hz$ für \nu_2 kaum über die Schärfe der theoretischen Werte hinausgehen.
Der Grund dafür, dass der Fit der arctan-Funktion die in (Abb. 4) zu sehenden großen Abweichungen zeigt, ist wahrscheinlich, dass keine Werte über $40 kHz$ genommen worden sind und somit die Asymptote vom Programm ohne Basis erzeugt werden muss.
