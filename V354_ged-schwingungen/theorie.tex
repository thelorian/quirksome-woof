% !TEX program = lualatex
\documentclass{scrartcl}
\usepackage{amsmath}
\usepackage{amssymb}
\usepackage{unicode-math}
\usepackage{mathtools}
\usepackage{fontspec}
\usepackage{polyglossia}
\begin{document}


    \section{Zielsetung}
    In diesem Versuch werden anhand eines LRC-Schwingkreises die verschieden Lösungen der Schwingungsgleichung untersucht.
    Es werden gemessen: der effektive Widerstand im Fall der homogenen gedämpften Schwingung,
    der Widerstand bei dem der aperiodische Grenzfall eintritt und inhomogenen Fall die Resonanzfrequenz, sowie die Phasenverschiebung,
    die der Schwingkreis verusacht.


    \section{Theorie}
    \subsection{Schwingungsgleichung}
    \label{sub:Schwingungsgleichung}

    \label{sec:Theorie}
    Der gegebene, wie auch der standard LRC-Reihenschwingkreis besteht aus Einer Induktivität L, einem ohmschen Widerstand R und einer Kapazität C.
    Im Idealfall ($R=0$) wird nur zwischen der Induktivität und der Kapazität harmonisch Energie ausgetauscht, ist $R > 0$ so nimmt diese auch exponentiell ab,
    da sie vom Widerstand des Stromkreises in Wärme umgesetzt wird.

    Aus der Maschenregel (2. kirchhoffsches Gesetz) folgt, dass die Summe der Spannungen in einem geschlossenen Stromkreis gleich Null ein müssen, also muss gelten:
    \begin{equation}
        U_L(t)+U_R(t)+U_C(t)=0.
        \label{eqn:Krichhoff2}
    \end{equation}
    Für die gegebenen Spannungen gelten die Identitäten
    \begin{align*}
        U_L &= \dot{I}L \\
        U_R &= IR  \\
        U_C &= \frac{Q}{C}.
    \end{align*}
    Dies ergibt mit $I=\dot{Q}$ die homogene Schwingungsgleichung in I:
    \begin{equation}
        L\ddot{I}+R\dot{I}+\frac{1}{C}I=0
        \label{eqn:Schwingungsgleichung}
    \end{equation}
    Die Lösung für diese DGL lautet
    \begin{equation}
        I(t)=e^{-2\pi\mu t}(A_0e^{2\pi\nu it}+A_1e^{-2\pi\nu it})
        \label{eqn:Fundamentalsys}
    \end{equation}
    mit
    \begin{align}
        \mu &= \frac{R}{4\pi L}\\
        \text{und}\\
        \nu &= \frac{1}{2\pi}\sqrt{\frac{1}{LC}-\frac{R^2}{4L^2}}
    \end{align}

    Der Verlauf der entsprechenden Ladungs- bzw. Spannungskurve hängt nun davon ab,
    ob $\nu$ reelwertig ist oder echt komplexe Werte annimmt:
    \begin{equation*}
        \nu \in \mathbb{R} \iff \frac{1}{LC} > \frac{R^2}{4L^2}.
    \end{equation*}

    Mit $e^{i\phi}= cos(\phi)+isin(\phi)$ und (\ref{eqn:Fundamentalsys}) folgt dann:
    \begin{equation}
        I(t) = e^{-2\pi\mu t}cos(2\pi\nu t)
    \end{equation}
    Die Zeit, zu der Exponent genau -1 ist nennt man Abklingdauer:
    \begin{equation}
        T_{ex} := \frac{1}{2\pi\mu}.
        \label{eqn:Abkling}
    \end{equation}
    Für $\nu = 0$ tritt der sog. aperiodische Grenzfall ein,
    wo $I(t)$ durch nur eine Exponentialfunktion beschrieben werden kann und es somit zu keiner Schwingung kommt.
    In diesem Fall nähert sich die Auslenkung am schnellsten Null von allen Lösungen der DGL.
    \subsubsection{Getriebene Schwingung}
    \label{subs:Getriebene Schwingung}

    Im weitern Verlauf des Versuchs wird ein mit einer Wechselspannung getriebener Schwingkreis
    untersucht. Die Differentialgleichung muss daher um  eine Inhomogenität der Form:
    \begin{equation*}
        U_0e^{i\omega_0 t}
    \end{equation*}
    ergänzt werden.Nachdem sich das System eingeschwungen hat und $\omega =\omega_0$ gilt berechnet sich die Spannung am Schwingkreis mit
    \begin{equation}
        |U(t)| = U_0\sqrt{\frac{(1-LC\omega^2)^2+\omega^2R^2C^2}{((1-LC\omega^2)^2+\omega^2R^2C^2)}}
    \end{equation}
    Mit der komplexen Phase
    \begin{equation}
        \label{eqn:Phase}
        \phi(\omega) = atan\left(\frac{-\omega RC}{1-LC\omega^2}\right).
    \end{equation}
    Da gilt $|U_C|=|U(t)|$, folgt
    \begin{equation}
        U_C(\omega)=\frac{U_0}{\sqrt{(1-LC\omega^2)^2+\omega^2R^2C^2}}
    \end{equation}
    An dieser Formel kann man erkennen, dass gilt
    \begin{align}
    {\omega &= 0 \implies U_C = U_0} ,

    \text{und außerdem, dass für große}
    \omega : U_C &\propto \frac{1}{\omega^2} \rightarrow 0.
    \end{align}
    Für Phasen $\Phi_1,\Phi_2$ bei  $\frac{\pi}{4}$ und $\frac{3}{4} \pi$ respektive gilt:
    \begin{equation}
        \label{eqn:Spektrum}
        \omega_{1,2}=\pm \frac{R}{2L}+\sqrt{\frac{R^2}{4L^2}+\frac{1}{LC}}
    \end{equation}
    Außerdem gilt für die Resonanzfrequenz
    \begin{equation}
        \label{eqn:ResFrequenz}
        \omega_{res} = \sqrt{\frac{1}{LC}-\frac{R^2}{2L^2}} ,
    \end{equation}
    wo der Maximalwer von $U_C(\omega)$ erreicht wird, der i.d.R. deutlich höher ist als U_0.

     Wenn die Dämpfung schwach ist, also gilt
     \begin{equation*}
         \frac{R^2}{2L^2}\ll \frac{1}{LC} ,
     \end{equation*}
     so vereinfacht sich das Problem auf eine ungedämpfte Schwingung, bei der die Eigenfrequenz des Schwingkreises dann gegeben ist durch
     \begin{equation}
         \omega_0 = \sqrt{\frac{1}{LC}} .
     \end{equation}

     Für diesen Fall gilt dann $\omega_{res} \approx \omeaga_0$ und daher
     \begin{equation}
         \label{eqn:Güte}
         U_C(\omega_{res})= qU_0
         \text { mit }
          q=\frac{1}{\omega_0 RC}
     \end{equation}
     Man nennt q dann die Güte des Schwingkreises. Zusätzlich wird gewöhnlich auch die Breite der Resonanzkurve von $U_C(\omega)$ angegeben
     mit der Näherung
     \begin{equation}
         \label{eqn:Breite}
         \omega_+ -\omega_- \approx \frac{R}{L} .
     \end{equation}

\end{document}
