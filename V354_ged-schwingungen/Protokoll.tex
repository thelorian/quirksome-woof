%!TEX program = lualatex
%\documentclass{scrartcl}
%\usepackage{amsmath}
%\usepackage{amssymb}
%\usepackage{unicode-math}
%\usepackage{mathtools}
%\usepackage{fontspec}
%\usepackage{polyglossia}
%\usepackage{siunitx}
%\usepackage{graphicx}
%\usepackage{scrhack}
%\usepackage{float}
%\floatplacement{figure}{htbp}
%\floatplacement{table}{htbp}
%\usepackage[section,below,]{placeins}
\input{header.tex}
\input{anleitung.bib}
\title{Versuch Nr. 354 Gedämpfte und erzwungene Schwingungen}
\author{Kyra Klos \and Leonard Borg}
\date{Durchführung: 3.11.2015; Abgabe: 10.11.2015}

\begin{document}
   \maketitle
   \section{Zielsetung}
   In diesem Versuch werden anhand eines LRC-Schwingkreises die verschieden Lösungen der Schwingungsgleichung untersucht.
   Es werden gemessen: der effektive Widerstand im Fall der homogenen gedämpften Schwingung,
   der Widerstand bei dem der aperiodische Grenzfall eintritt und inhomogenen Fall die Resonanzfrequenz, sowie die Phasenverschiebung,
   die der Schwingkreis verusacht.

   \section{Theorie}
   \label{sec:Theorie}
   \subsection{Schwingungsgleichung}
   \label{sub:Schwingungsgleichung}

   Der gegebene, wie auch der standard LRC-Reihenschwingkreis besteht aus Einer Induktivität L einem Ohmschen Widerstand R und einer Kapazität C.
   Im Idealfall (R=0) wird nur zwischen der Induktivität und der Kapazität harmonisch Energie ausgetauscht, ist R>0 so nimmtdiese auch exponentiell ab,
   da sie vom Widerstand des Stromkreises in Wärme umgesetzt wird.
       Da sich alle Spannungen in einem geschlossenen einfachen Stromkreis dank der Maschenregel zu null addieren müssen,
   ergibt sich die Gleichung:
       \begin{equation}
           U_L(t)+U_R(t)+U_C(t)=0.
           \label{eqn:Krichhoff2}
       \end{equation}
       Für die gegebenen Spannungen gelten die Identitäten
       \begin{align*}
           U_L &= \dot{I}L \\
           U_R &= IR  \\
           U_C &= \frac{I}{C}.
       \end{align*}
       Dies ergibt mit $I=\dot{I}$ die homogene Schwingungsgleichung in I:
       \begin{equation}
           L\ddot{I}+R\dot{I}+\frac{1}{C}I=0
           \label{eqn:Schwingungsgleichung}
       \end{equation}
       Die Lösung für diese DGL lautet
       \begin{equation}
           I(t)=e^{-2\pi\mu t}(A_0e^{2\pi\nu it}+A_1e^{-2\pi\nu it})
           \label{eqn:Fundamentalsys}
       \end{equation}
       mit physikalischen Konstanten
       \begin{align}
           \mu &= \frac{R}{4\pi L}\\
           \nu &= \frac{1}{2\pi}\sqrt{\frac{1}{LC}-\frac{R^2}{4L^2}}
       \end{align}

       Der Verlauf der entsprechenden Ladungs- bzw. Spannungskurve hängt nun davon ab,
       ob $\nu$ reelwertig ist oder echt komplexe Werte annimmt:
       \begin{equation*}
           \nu \in \mathbb{R} \iff \frac{1}{LC} > \frac{R^2}{4L^2}
       \end{equation*}

       Mit $e^i\phi= cos(\phi)+isin(\phi)$ und (\ref{eqn:Fundamentalsys}) folgt dann:
       \begin{equation}
           I(t) = e^{-2\pi\mu t}cos(2\pi\nu t)
       \end{equation}

       Die Zeit, zu der Exponent genau -1 ist nennt man Abklingdauer:
       \begin{equation}
           \label{eqn:Abkling}
           T_{ex} := \frac{1}{2\pi\mu}.
       \end{equation}

       Für $\nu = 0$ tritt der sog. aperiodische Grenzfall ein,
       wo $I(t)$ durch nur eine Exponentialfunktion beschrieben werden kann und es somit zu garkeiner Schwingung kommt.
       In diesem Fall nähert sich die Auslenkung am schnellsten Null von allen Lösungen der DGL.

       \subsubsection{Getriebene Schwingung}
       \label{subs:Getriebene Schwingung}
       Im weitern Verlauf des Versuchs wird ein mit einer Wechselspannung getriebener Schwingkreis
       untersucht. Die Differentialgleichung dadurch eine Inhomogenität der Form:
       \begin{equation*}
           U_0e^{i\omega_0 t}
       \end{equation*}
       Nachdem sich das System eingeschwungen hat und $\omega = \omega_0$ gilt
       berechnet sich die Spannung am Schwingkreis mit
       \begin{equation}
           |U(t)| = U_0\sqrt{\frac{(1-LC\omega^2)^2+\omega^2R^2C^2}{((1-LC\omega^2)^2+\omega^2R^2C^2)}}
       \end{equation}
       Mit der komplexen Phase
       \begin{equation}
           \label{eqn:Phase}
           \phi(\omega) = atan\left(\frac{-\omega RC}{1-LC\omega^2}\right)
       \end{equation}
       Da $|U_C|=|U(t)|$ folgt
       \begin{equation}
           U_C(\omega)=\frac{U_0}{\sqrt{(1-LC\omega^2)^2+\omega^2R^2C^2}}
       \end{equation}
       An dieser Formel kann man erkennen, dass für $\omega = 0 U_C = U_0$ gilt,
       und außerdem, dass für große $\omega : U_C \propto \frac{1}{\omega^2} \rightarrow 0.$
       Für Phasen $\Phi_1,\Phi_2$ bei  $\frac{\pi}{4}$ und $\frac{3\pi}{4}$ respektive
       gilt:
       \begin{equation}
           \label{eqn:Spektrum}
           \omega_{1,2}=\pm \frac{R}{2L}+\sqrt{\frac{R^2}{4L^2}+\frac{1}{LC}}
       \end{equation}
       Außerdem gilt für die Resonanzfrequenz
       \begin{equation}
           \label{eqn:ResFrequenz}
           \omega_{res} = \sqrt{\frac{1}{LC}-\frac{R^2}{2L^2}},
       \end{equation}
       wo der Maximalwer von $U_C(\omega)$ erreicht wird, der i.d.R. deutlich höher ist als U_0.

        Wenn die Dämpfung schwach ist, also gilt
        \begin{equation*}
            \frac{R^2}{2L^2}\ll \frac{1}{LC},
        \end{equation*}
        so vereinfacht man das problem aus eine ungedämpfte Schwingung, bei der die Eigenfrequenz des Schwingkreises danngegeben ist durch
        \begin{equation}
            \omega_0 = \sqrt{\frac{1}{LC}}
        \end{equation}

        Für diesen Fall gilt dann $\omega_{res} \approx \omeaga_0$ und daher
        \begin{equation}
            \label{eqn:Güte}
            U_C(\omega_{res})= qU_0
            \text { mit }
             q=\frac{1}{\omega_0 RC}
        \end{equation}
        Man nennt q dann die Güte des Schwingkreises. Zusätzlich wird gewöhnlich auch die Breite der Resonanzkurve von $U_C(\omega)$ angegeben
        mit der Näherung
        \begin{equation}
            \label{eqn:Breite}
            \omega_+ -\omega_- \approx \frac{R}{L}
        \end{equation}

    \section{Durchführung}
    \label{sec:Durchführung}
    Als erstes wird an den gegebenen Schwingkreis mit vorab bekannten Größen L,R,C
    eine Nadelpulsspannung angelegt und parallel zum Kondensator wird ein Oszilloskop angeschlossen.
    Nun stellt man die Frequenz des Nadelpuls' so ein, dass man nach dem jeweiligen Puls Relaxionsverhalten beobachten kann
    (die Spannung sollte mindestens um $\frac{1}{e}$ abgenommen haben).
    Von der enstehenden Abbildung einer gedäpften (Co-)Sinusschwingung wird der Thermodruck genommen,
    und bei den Extrema der Funktion werden jeweils Zeit seit dem Nadelpuls und Amplitude gemessen.

    Um den Grenzwiderstand $R_{ap}$ für den aperiodischen Grenzfall zu bestimmen wird nun,
    anstatt einem statischen R ein regelbarer Widerstand in den Schwingkreis geschaltet.
    Dieser wird von seinem Maximalwert so weit reduziert, dass sich eine sichtbare Überschwingung ausbildet
    und dann zurück auf eben den Wert eingestellt wo das Oszilloskop das erste mal eindeutig dem relaxionsverhalten eines geladenen Kondensators gleicht
    Der an diesem Punkt abzulesende Widerstand wird notiert.

    Zur Messung des Verhaltens des getriebenen Schwingkreises wird eine Wechselspannung and den Schwingkreis gelegt
    und zum Abgleich auf den zweiten Kanal des Oszilloskops gespeist.
    Als erstes wird die Amplitude der angelegten Wechselspannung aufgenommen
    und im Folgenden wird die Frequenz $\nu$ am Sinusgenerator variiert.
    Es sollte darauf geachtet werden, dass die Messabstände bei stärkeren Änderungen entsprechend kleiner zu wählen sind.

    Als Abschluss wird $\Phi(\omega)$ bestimmt die Phasenverschiebung, die der Schwingkreis aufgrund seiner Impedanz hervorruft.
    Zunächst wird die Erregerfrequenz klein gehalten, sodass die Erregung überwiegt und beide Wellen phasengleich sind
    und justiert das Oszilloskop so, dass bei beiden Wellen die Nulldurchgänge im selben punkt liegen.
    Und nun variiert man wie zuvor die Frequenz und misst die entstehenden Abstände zwischen den Nulldurchgängen.

    \section{Auswertung}
    \label{sec:Auswertung}
    Hier folgt nun die Auswertung des Experiments, das mit Hilfe des Gräts $2$ durchgeführt wurde.
    Die Plots wurden mit Hilfe des Programs Gnuplot erstellt.
    \subsection[underline]{Zeitabhängigkeit der gedämpften Schwinungsamplitude}
    Zuerst wurde die abklingenden Amplituden $U_C$ mit dem Fehler $\pm{0,08}$$\si{\volt}$ des LCR-Schwingkreises im Bezug auf
    die Zeit $t_i$ mit der Ungenauigkeit von $\pm{2}$$\si{\micro\s}$ gemessen.(Siehe Abbildung \ref{fig:thermodruck})
    \begin{table}
      \centering
      \caption{Hier sieht man deutlich den exponentiellen Abfall der Ampiltuden der Schwingung mit $\SI{1997}{\hertz}$ Eregerfrequenz zu der Zeit}
      \label{tab:Daten1}
      \begin{tabular}{S S}
       \toprule
       {$U_C\:/\:\si{\volt}$} & {$t_i\:/\:\si{\micro\sec}$} \\
       \midrule
      4,56 & 12\\
      3,76 & 28\\
      3,36 & 44\\
      2,64 & 56\\
      2,40 & 70\\
      2,24 & 86\\
      1,76 & 102\\
      1,60 & 116\\
      1,36 & 128\\
      1,20 & 146\\
      0,96 & 158\\
      0,96 & 173\\
      0,80 & 186\\
      0,72 & 202\\
      0,56 & 216\\
      0,56 & 230\\
      0,48 & 240\\
       \bottomrule
      \end{tabular}
     \end{table}
    Besonders hervor zu heben ist der Dämpfungswiderstand des aperiodischen Grenzfalls,der experimentel einen Wert von $\SI{3,19(1)}{\kilo\ohm}$ beträgt.
    Verglichen mit dem durch () errechneten Widerstand von $\SI{4,39(9)}{\kilo\ohm}$, ist eine Differenz von ~ $\SI{1,4}{\kilo\ohm}$ zu ersehen.
    Durch die Ausgleichrechnung () erhält man $\mu$ aus dem Exponenten der Gleichung () und kann somit folgendes berechnen:
    \begin{align*}
     T_{ex} &= \si{123(1)e-6}{\sec}\\
     R_{eff} &= \SI{123(1)}{\ohm}
    \end{align*}
    Vergleicht man nun den effektiven Widerstand $R_{eff}$ mit dem in der Schaltung eingesetzten von $\SI{48,1(1)}{\ohm}$ , ist eine Diskrepanz von ungefähr $\SI{50}{\ohm}$ zu erkennen.
    \begin{figure}
    \centering
    \includegraphics{Thermo.jpg}
    \caption{Thermodruck des zeitlichen Abklingens der gedämpften Schwingungsamplitude\\ mit der Einfüllenden(in grün).}
    \label{fig:thermodruck}
    \end{figure}
    \newpage
    \subsection[underline]{Messung der Frequenzabhängigkeiten an einem LCR-Schwingkreis}
    In der folgenden Tabelle sind die Daten aus der Messreihe der Kondensatorspannung  mit der Ungenauigkeit $\pm{0,3}$$\si{\volt}$ zu den Frequenzenmit mit der Ungenauigkeit $\pm{0,03}$$\si{\kilo\hertz}$, sowie das Verhältnis dieser zur Eregerspannung von ca. $\SI{10}{\volt}$ zu finden.
    Diese Daten sind in zwei Plots einem in linearel (\ref{fig:Linear1}) und einem in halblogarithmischer (\ref{fig:Halb1}) Form aufgetragen.
    Wichtig hierbei zu erwähnen ist,dass ab einschließlich dem Wert $\SI{27,15}{\kilo\hertz}$ die Skala vergrößert wurde, sodass sich die darauffolgenden Werte leicht verschieben können.
    Aus dem Linearen der darauffolgenden Plots, in dem die Frequenz zum Spannungsverhältnis aufgetragen wurde, sind die experimentellen Werte für die Resonanzüberhöhung bzw. Güte q, sowie die Breite der Resonanzkurve $\nu_1 - \nu_2$ entnommem und werden hier mit den durch () berechneten Werten mit einem Gesamtwiderstand von $\SI{559,5(5)}{\ohm}$ verglichen:
    \begin{align*}
    q_{theo} &= \si{3,923(5)} & q_{exp} &= \si{3,72}\\
    \text{Theoretisch:} \symup{\nu_+} - \symup{\nu_-} &= \SI{8,81(3)e3}{\hertz} & \text{Experimentell:} \symup{\nu_+} - \symup{\nu_-} &= \SI{11,19e3}{\hertz}
    \end{align*}
    Die realtiv gerringen Differnezen von q bei etwa $0,2$ und bei Resonanzkurvenbreite bei etwa $\SI{1,3e3}{\ohm}$ sind im Fehlertolerenazbereich und benötigen somit keine weitere Erläuterung.
    \begin{table}
      \centering
      \caption{Die Messdaten zeigen hierbei einen deutlichen Anstieg bis zum Maximum, sowie eine nachträglichen Abfall gleichermaßen}
      \label{tab:Daten2}
      \begin{tabular}{S S S}
        \toprule
        \multicolumn{1}{c}{$\si{\nu}\:/\:\si{\kilo\hertz}$} & \multicolumn{1}{c}{$U_C\:/\:\si{\volt}$} & \multicolumn{1}{c}{$U_C/U$}\\
        \midrule
        2,00 & 10,0 & 1,00\\
        4,00 & 10,0 & 1,00\\
        5,00 & 10,2 & 1,02\\
        7,50 & 10,4 & 1,04\\
        9,00 & 10,6 & 1,06\\
       11,00 & 11,0 & 1,10\\
       14,00 & 11,5 & 1,15\\
       16,00 & 12,0 & 1,20\\
       18,00 & 12,6 & 1,26\\
       18,65 & 13,0 & 1,30\\
       19,50 & 13,6 & 1,36\\
       20,30 & 14,0 & 1,40\\
       21,00 & 14,6 & 1,46\\
       21,45 & 15,0 & 1,50\\
       22,50 & 15,8 & 1,58\\
       23,00 & 16,3 & 1,63\\
       23,60 & 16,9 & 1,69\\
       24,20 & 17,6 & 1,76\\
       24,55 & 18,0 & 1,80\\
       25,00 & 18,6 & 1,86\\
       25,35 & 19,0 & 1,90\\
       25,65 & 19,4 & 1,94\\
       26,10 & 20,1 & 2,01\\
       26,50 & 20,7 & 2,07\\
       26,80 & 21,2 & 2,12\\
       27,10 & 21,7 & 2,17\\
       27,15 & 22,4 & 2,24\\
       27,50 & 22,8 & 2,28\\
       28,00 & 24,0 & 2,40\\
       28,40 & 24,8 & 2,48\\
       28,60 & 25,2 & 2,52\\
       29,05 & 26,4 & 2,64\\
       29,35 & 27,2 & 2,72\\
       29,50 & 27,6 & 2,76\\
       29,85 & 28,4 & 2,84\\
       30,00 & 28,8 & 2,88\\
       30,15 & 29,2 & 2,92\\
       30,25 & 29,6 & 2,96\\
       30,45 & 30,0 & 3,00\\
       30,55 & 30,4 & 3,04\\
       30,70 & 30,8 & 3,08\\
       30,85 & 31,2 & 3,12\\
       31,00 & 31,6 & 3,16\\
       31,10 & 32,0 & 3,20\\
       31,25 & 32,4 & 3,24\\
       31,55 & 33,4 & 3,34\\
       31,70 & 33,2 & 3,32\\
       31,90 & 34,0 & 3,40\\
       32,20 & 34,8 & 3,48\\
       32,35 & 35,2 & 3,52\\
       32,55 & 35,6 & 3,56\\
       32,80 & 36,0 & 3,60\\
       33,35 & 36,8 & 3,68\\
       34,00 & 37,2 & 3,72\\
       34,50 & 36,8 & 3,68\\
       35,00 & 36,4 & 3,64\\
       35,50 & 35,2 & 3,52\\
       36,00 & 33,6 & 3,36\\
       36,50 & 32,3 & 3,23\\
       37,00 & 30,6 & 3,06\\
       37,50 & 29,0 & 2,90\\
       38,00 & 27,2 & 2,72\\
       38,50 & 25,6 & 2,56\\
       39,00 & 24,0 & 2,40\\
       39,50 & 22,4 & 2,24\\
       40,00 & 21,2 & 2,12\\
       41,00 & 19,0 & 1,90\\
       42,00 & 16,8 & 1,68\\
       43,00 & 15,2 & 1,52\\
       44,00 & 14,0 & 1,40\\
       45,00 & 12,8 & 1,28\\
       46,00 & 11,6 & 1,16\\
       47,00 & 10,8 & 1,08\\
       48,00 & 10,0 & 1,00\\
       49,00 & 9,2 & 0,92\\
       50,00 & 8,6 & 0,86\\
       55,00 & 6,4 & 0,64\\
    \bottomrule
    \end{tabular}
    \end{table}
    \begin{figure}
      \centering
      \includegraphics[width=\textwidth]{Linear1.jpeg}
    \caption{Aus der hierzusehenden linearen Darstellung der Freuquenzabhänigkeit der Spannung \\ sind die Daten zur Güte und die Resonanzkurve mit ihren eingezeichten Ränder anzulesen.}
    \label{fig:Linear1}
    \end{figure}
    \begin{figure}
      \centering
      \includegraphics[width=\textwidth]{Halblog1.jpeg}
      \caption{Hier ist die halblogarithmische Darstellung des Spannungs-Frequenzverhältnisses zu sehen.}
      \label{fig:Halb1}
    \end{figure}

    \subsection[underline]{Messung der Phasen zwischen Erreger- und Kondensatorspannung}
    Im letzten Teil des Experiments wurden die Phasendifferenzen $\delta$t zu den einzelnen Frquenzen gemessen und daraus die Pahsenverschiebuung $\varphi$ im Bogenmaß berechnet.
    Hier sind die Phasenverschiebung $\varphi$ und die Phasendifferenz $\delta$t mit der Ungenauigkeit $\pm{0,2}$$\si{\micro\s}$ zur Frequenz $\nu$ vom Verhälnis LCR-Schwingkreis zu Sinus-Schwingungsgenerator.
    \begin{table}
      \centering
      \caption{Hier sieht man die Werte, der durch Lissajou-Figuren gemessenen Phasendifferenz, und die dazugehörige Verschiebung zu den Frequenzen.}
      \label{tab:Daten3}
      \begin{tabular}{S S S}
        \toprule
        \multicolumn{1}{c}{$\symup{\nu}\:/\:\si{\kilo\hertz}$} & \multicolumn{1}{c}{$\symup{\Delta}t\:/\:\si{\micro\sec}$} & \multicolumn{1}{c}{$\symup{\varphi}\:/\:rad$}\\
        \midrule
        2,0 & 0,00 & 0,0000 \\
        5,0 & 1,00 & 0,0314 \\
       10,0 & 1,00 & 0,0628 \\
       15,0 & 1,50 & 0,1414 \\
       20,0 & 1,80 & 0,2262 \\
       25,0 & 2,10 & 0,3300 \\
       27,5 & 3,00 & 0,5184 \\
       30,0 & 4,00 & 0,7540 \\
       30,5 & 4,10 & 0,7854 \\
       31,0 & 4,50 & 0,8765 \\
       32,0 & 5,20 & 1,0455 \\
       33,0 & 6,00 & 1,2440 \\
       33,5 & 6,50 & 1,3681 \\
       34,0 & 6,70 & 1,4313 \\
       34,5 & 7,50 & 1,6258 \\
       35,0 & 7,50 & 1,6493 \\
       35,5 & 8,00 & 1,7844 \\
       36,0 & 8,50 & 1,9227 \\
       36,5 & 8,70 & 1,9952 \\
       37,0 & 9,00 & 2,0923 \\
       37,5 & 9,25 & 2,1795 \\
       38,0 & 9,50 & 2,2682 \\
       38,5 & 9,50 & 2,2980 \\
       39,0 & 9,50 & 2,3279 \\
       39,5 & 9,75 & 2,4198 \\
       40,0 & 9,75 & 2,4504 \\
       \bottomrule
     \end{tabular}
     \end{table}

    Die Plots der Frequenzen in Abhängigkeit der Phasenverschiebung sind in halblogarithmischer (\ref{fig:Halb2}) und linearer (\ref{fig:Linear2}) Form in den folgenden Abbildungen zusehen.\\
    Aus Letztrem werden die experimentellen Werte für die Resonanzspannung $\nu_res$ , die sich im Bereich von $\pi/2$ befindet, sowie die Ränder der Resonanzkurve $\nu_1$ und $\nu_2$ bei $\pi/4$ und $3*\pi/2$ entnommen.
    \begin{figure}
      \centering
      \includegraphics[width=\textwidth]{Linear2.jpeg}
      \caption{Die lineare Darstellung des Verhältnisses Phasenverschiebung von Erreger- und Kondensatorspannung zur Frequenz.}
      \label{fig:Linear2}
    \end{figure}
    Hier in der linearen Abbildung ist gut der extreme Antsieg der arctan-annäherden Kurve zu sehen.\\Die Spitze und ihr Abflachen sind, durch die begrentzte Maximalfrequenz nur zu erahnen.
    \begin{figure}
      \centering
      \includegraphics[width=\textwidth]{Halblog2.jpeg}
      \caption{Die halblogarithmische Darstellung der Ferquenzabhängigkeit der Phase zwischen Erreger- und Kondensatorspannung.}
      \label{fig:Halb2}
    \end{figure}
    \newpage
    Die experimentellen Werte werden mit den durch (\ref{eqn:ResFrequenz}) und (\ref{eq}) berechneten Werten verglichen:
    \begin{align*}
      \text{Theoretisch:}\nu_{res} &= \SI{34,55(8)}{\kilo\hertz} & \nu_1 &= \SI{30,78(4)}{\kilo\hertz} & \nu_2 &= \SI{38,80(3)}{\kilo\hertz}\\
      \text{Experimentell:}\nu_{res} &= \SI{34,60}{\kilo\hertz} & \nu_1 &= \SI{30,50}{\kilo\hertz} & \nu_2 &= \SI{39,0}{\kilo\hertz}
    \end{align*}

    \section{Disskusion}
    \label{sec:Disskusion}
  Zum Abschluss werden nun Auffälligkeiten und Unterschiede von theoretischen und experimentellen Werten nähere untersucht.
  Der Unterschied vom erechneten effektiven Widerstand $R_{eff}$ und dem im Schaltkreis eingesetzten beträgt, wenn man kleine Messungenauigkeiten vernachlässigt
  etwa $\SI{55}{\ohm}$, was ungefähr der Wert der des Innenwiderstandes des Funktionengenerators ist und somit geklärt wäre.
  Der nächste zu betrachtende Wert ist $R_{ap}$ der Widerstand des aperiodischen Grenzfalls, hier ist ein Unterschied von etwa $\SI{1,4}{\ohm}$ zu vermerken, dieser rührt daher,
  %dass in der Rechnung der Widerstand anderer Bauteile vernachlässigt werden (z.B. Spule) und von der experimentellen Seite ein genaues Einstellen und Ablesen nicht sicher gegeben werden konnte.
  Die folgende Werte zur Güte sowie zur Breite der Resonazkurve beim Untersuchen der Frequnezabhängigkeit der Kondensatorspannung, liegt klar im Fehlertolerenzbereich.
  Bei der Phasenuntersuchungen zeigen sich nur sehr gerringe Diskrepanzen zwischen den berechnten und gemessenen Werten, was besonders durch Insturmente, wie dem digitalen Oszilloskop und dessen Curser für genaue Messungen, gegeben ist.
\printbibliography
\end{document}
