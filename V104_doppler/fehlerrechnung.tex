%!TEX program = lualatex
\documentclass{scrartcl}
\usepackage{amsmath}
\usepackage{amssymb}
\usepackage{unicode-math}
\usepackage{mathtools}
\usepackage{fontspec}
\usepackage{polyglossia}



\begin{document}


\section{Fehlerrechnung}
\label{sec:Fehlerrechnung}

Alle Mittelwerte, sofern nicht anders vermerkt werden mit dem arithmetischen Mittel:
\begin{equation}
    \overline{x}  = \frac{1}{N} \sum_{i=1}^N x_i
\end{equation}
berechnet.\\
Der statistische Fehler dieses Wertes wird berechnet mit
\begin{equation}
    \Delta\overline{x} = \frac{1}{\sqrt{N}}\sqrt{\frac{1}{N-1}\sum_{i=1}^N (x_i-\overline{x})^2}}.
\end{equation}
Werden Fehlergrößen zur weiteren Berechnung genutzt, so wird der Fehler im Resultat
mit der Gauß'schen Fehlerfortpflanzung berechnet:
\begin{equation}
\Delta f = \sqrt{\sum_{i=1}^N \left(\frac{\partial f}{\partial x_i}\right)^2(\Del{x_i})}
\end{equation}

Auftretende Ausgleichsgraden werden berechnet nach der linearen Regression mit
\begin{align}
    y &= ax+b\\
    a &= \frac{\overline{xy}-\overline{x}\overline{y}}{\overline{x^2}-\overline{x}^2}\\
    b &= \frac{\overline{x^2}\overline{y}-\overline{y}\overline{xy}}{\overline{x^2}-\overline{x}^2}
\end{align}


\end{document}
