\section{Diskussion}
\label{sec:Diskussion}

Die beiden Ergebnisse der Messungen für $D$, $D_{\text{dyn}} =  \num{2,05(0,03)e-2}\si{\newton\meter\squared}$ und $D_{\text{stat}} = (0,0308 \pm 0,0061)\SI{\newton \meter}$
unterscheiden sich um den Faktor $ 1,5.$ Die Ursache dafür ist vermutlich die grobe Ungenauigkeit der statischen Messung, bei der sowohl das Ablesen des Winkels und der Kraft, als auch die genaue senkrechte Stellung einzuhalten, einige Schwierigkeiten verusachte.
Auch ist die Metallstange, die zur Bestimmung von $I_{\text{D}}$ vernachtlässigt wurde zwar im Vergleich zu den Zylindern leicht, macht aber noch etwa 5\% des Trägheitsmonets aus.


Die Abweichung um Faktor $3$ für beide einfachen Körper lässt auf einen schwerwiegenden systematsichen Fehler schließen.
Das für die Puppe berechnete theoretische Trägheitsmoment weicht immens von den Werten ab,
die die Messungen ergeben haben, 
aber der Fehler von mitunter über $100\%$ Abweichung sowie das Ergebnis, 
dass Pose1 und Pose2 das gleiche Trägheitsmoment haben sollten, 
während ihre Messwerte weit auseinander liegen, legt nahe, dass zur Berechnung zu viele Vereinfachende Annahmen getroffen wurden.
