\section{Diskussion}
\label{sec:Diskussion}

Die beiden Ergebnisse der Messungen für D, $D_{\text{dyn}} =  \num{2,05e-2}\si{\newton\meter\squared}$ und $D = (0,0308 \pm 0,0061)\SI{\newton \meter}$
unterscheiden sich um den Faktor $ 1,5.$ Die Ursache dafür ist vermutlich die grobe Ungenauigkeit der statischen Messung, bei der sowohl das Ablesen des Winkels und der Kraft, als auch die genaue senkrechte Stellung einzuhalten einige Schwierigkeiten verusachte.
Auch ist die Metallstange die zur bestimmung von $I_D$ vernachtlässigt wurde zwar im Vergleich zu den Zylindern leicht macht aber noch etwa 5\% des Trägheitsmonets aus.


Die Abweichung um Faktor $3$ für beide einfachen Körper lässt auf einen Schwerwiegenden systematsichen Fehler schließen,
möglicherweise auch ein Rechenfehler etwa ein Faktor $\pi$.
Das für die Puppe berechnete Trägheitsmoment stimmt für Pose 2 gut mit den Messwerten überein, doch der große Unterschied zum Trägheitsmoent in Pose 1 legt nahe,
dass die Annahme, dass beide Trägheitsmomente gleich sind on einm zu zylinderförmigen Torso ausgeht.
