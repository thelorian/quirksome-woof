\section{Auswertung}
\label{sec:Auswertung}

\subsection{Bestimmung der Gerätekonstanten}
\label{sub:subsection label}

Als erstes, bevor mit dem gegebenen Gerät gemessen werden kann, müssen die geräteigenen Größen 
der Winkelrichtgröße $D$ und des Eigentträgheitsmoments $I_{text{D}}$ bestimmt werden.
Dafür wurde zunächst der Abstand 
\begin{align*}
  r = \num{16,95(0,1)}\si{\milli\metre}
\end{align*} 
des betrachteten Punktes von der Drehachse gemessen.
Der von dem an diesem Punkt senkrecht aufliegenden Federkraftmesser (senkrecht aus dem Grund, dass die skalare Formel des Kreuzproduktes von $M$  verwendet werden kann)
erzwungene Winkel wird zusammen mit der abgelesenen Kraft mithilfe von (4) in die Winkelrichtgröße verrechnet:
\begin{table}
\centering
\caption{Kraft bei Auslenkung um $\varphi$ und errechnetes $D$.}
\begin{tabular}{ S[table-format=1,3] S[table-format=1,3] S[table-format=1,4]}
\toprule
$\varphi\:/\:\text{rad}$ & $F\:/\:\si{\newton}$ & $D_{\text{stat}}\:/\:\si{\newton \metre}$\\
\midrule

0,349  &  0,095 & 0,0461\\
0,785  &  0,160 & 0,0345\\
1,047  &  0,190 & 0,0307\\
1,396  &  0,235 & 0,0285\\
1,570  &  0,260 & 0,0281\\
1,745  &  0,280 & 0,0272\\
2,094  &  0,320 & 0,0259\\
2,269  &  0,40  & 0,0299\\
2,618  &  0,45  & 0,0291\\
3,142  &  0,51  & 0,0275\\
\bottomrule
\end{tabular}
\end{table}

Der Mittelwert dieser berechneten Werte ergibt 
\begin{align*}
  D_{\text{stat}} = \num{3,08(0,61)e-2}\si{\newton\meter}.
\end{align*}

Um nun noch das Trägheitsmoment zu bestimmen,
wird $T^2$ gegen $a^2$ aufgetragen (siehe Abb. ) und der Ordinaten-Abschnitt mit linearer Regression berechnet. 
Es ergeben sich die Werte 
\begin{align*}
m &= \num{8,57(0,13)e-4} \si{\second\squared\per\milli\metre\squared} \\
b &=\num{4.1(0,4)} \si{\second\squared}\\
\text{mit}
T^2 &= m \cdot a^2 + b.
\end{align*}
Es wurden zwei Metallzylinder mit den Maßen 
\begin{align*}
  l &= \num{29,98(0,25)}\si{\milli\metre}\\
  r &= \num{17,63(0,28)}\si{\milli\metre} \\
  m_{\text{Zyl}} &= \num{222,60(1,18)}\si{\gram}
\end{align*}
 zur Messung verwendet.
\begin{table}
\centering
\caption{Periodendauern $T$ in Abhängigkeit vom Abstand $a$.}

\begin{tabular}{ S[table-format=3,2] S[table-format=1,3] S[table-format=5,0] S[table-format=2,3]}
\toprule
$a\:/\:\si{\milli\metre}$ & $T\:/\:\si{\second}$ &$a^2\:/\:\si{\milli\metre\squared}$&$T^2\:/\:\si{\second\squared}$\\
\midrule
54,1   & 2,652  &  2926  & 7,033 \\
77,49  & 3,012 &  6004  & 9,072 \\
95,64  & 3,378 &  9147  & 11,410\\
114,2  & 3,848 &  13041 & 14,807\\
134,2  & 4,258 &  18009 & 18,130\\
146,89 & 4,832 &  21576 & 23,348\\
164,14 & 5,294 &  26941 & 28,026\\
185,99 & 5,842 &  34592 & 34,129\\
205,49 & 6,4   &  42226 & 40,96\\
265,49 & 7,99  &  70484 & 63,840\\
\bottomrule
\end{tabular}
\end{table}

\begin{figure}
  \centering
  \includegraphics[width=\textwidth]{figure.pdf}
  \caption{Plot zur linearen Regression von $a^2$ zu $T^2$.}
\end{figure}


Mit den Formeln für das Trägheitsmoment des Zylinders, dem steinerschen Satz 
und (2) erhält man also für $D_{\text{dyn}}$ und $I_D$
\begin{align*}
    T &= 
    2\pi \cdot \sqrt{\frac{1}{D_{\text{dyn}}}(I_D+2I_\text{Zyl})}\\
    &=
    2\pi \cdot \sqrt{\frac{1}{D_{\text{dyn}}}(I_D+2 m_\text{Zyl}(\frac{r^2}{4}+\frac{l^2}{12})+2 m_\text{Zyl}a^2)}\\
    T^2 &=
    4\pi^2 \cdot 2\frac{m\text{Zyl}} {D_{\text{dyn}}}a^2 + 4\pi^2 \cdot \frac{1}{D_{\text{dyn}}}(I_D+2 m_\text{Zyl}(\frac{r^2}{4}+\frac{l^2}{12}))\\
    
    \implies b &= 4\pi^2 \cdot \frac{1}{D_{\text{dyn}}}\cdot(I_D+2 m_\text{Zyl}(\frac{r^2}{4}+\frac{l^2}{12}))\\
    m &= 8\pi^2\cdot\frac{m_\text{Zyl}}{D_\text{dyn}}\\
    \iff I_D 
    &= \frac{bD_{\text{dyn}}}{4\pi^2}-2(\frac{r^2}{4}+\frac{l^2}{12})\\
    D_{\text{dyn}}
    &= 8\pi \cdot \frac{m_{\text{Zyl}}}{m}.
    
    \leftarrow D_{\text{dyn}} &= 8\pi^2\frac{m_{\text{Zyl}}}{m} = \num{2,05(0,033)e-2}\si{\newton\meter\squared} \\    
    \text{und}\\
    I_D &= \frac{bD_{\text{dyn}}}{4\pi^2}-2(\frac{r^2}{4}+\frac{l^2}{12}) = \num{1,81(0,21)e-3} \si{\kilogram \meter \squared}
\end{align*}


\subsubsection{Bestimmung des Trägheitsmoments einfacher Körper}
\label{subs:Bestimmung des Trägheitsmoents einfacher Körper}

Nun werden die Trägheitsmomente von einfachen Körpern mit der dynamischen Methode bestimmt.
Die Maße der Kugel wurden bestimmt als 
\begin{align*}
  r &= \num{13,743(0,034)}\si{\centi\metre}  \\
  m &= \num{812,5}\si{\gram}, \\
  \text{die des Zylinders als}\\
  r &= \num{8,118(0,074)}\si{\centi\metre} \\
  l &= \num{14,183(0,076)}\si{\centi\metre}\\
  m &= \num{1973,8}\si{\gram}.
\end{align*}
Das Trägheitsmoment wird bestimmt aus den gemessenen Umlaufzeiten (Tabelle 3)
\begin{table}
  \centering
  \caption{Periodendauern $T$ für Kugel und Zylinder.}
  \begin{tabular}{c S[table-format=1,3] S[table-format=1,3]}
    \toprule
    $Messung $  & $T_{\text{Kugel}}\:/\:s$ &$T_{\text{Zylinder}}\:/\: $ \\
    \midrule
    1&    1,754 &     1,686\\
    2&    1,636 &     1,666\\
    3&    1,626 &     1,692\\
    4&    1,652 &     1,692\\
    5&    1,63  &     1,706\\
    6&    1,63  &     1,686\\
    7&    1,664 &     1,68 \\
    \midrule
    Mittelwert & 1,656  &1,687
    Fehler &   0,045    &0,012
    \bottomrule
  \end{tabular}
\end{table}

Es folgt aus ( ) für die Trägheitmomente mit $D$ aus der dynamischen Methode:
\begin{align*}
    I_\text{Kugel,dyn} &= \num{3,23(0,23)e-3}\si{\kilogram\metre\squared} \\
    I_\text{Zylinder,dyn} &= \num{3,29(0,22)}\si{\kilogram\metre\squared}\\
    \text{und $D$ aus der statischen:}\\
    I_\text{Kugel,stat} &= \num{3,9(0,5)e-3}\si{\kilogram\metre\squared} \\
    I_\text{Zylinder,stat} &= \num{4,0(0,5)}\si{\kilogram\metre\squared}.\\
\end{align*}

Die theoretischen Werte für diese Träghitsmomente ergeben sich aus den Formeln für die einfachen Trägheitsmomente durch den Schwerpunkt:
\begin{align*}
    I_\text{Kugel,theo} &= \frac{2}{5}mr^2 = \num{6,138(0,03)e-3}\si{\kilogram\metre\squared} \\
    I_\text{Zylinder,theo} &= \frac{1}{2}mr^2 = \num{6,50(0,12)e-3}\si{\kilogram\metre\squared}.
\end{align*}

\subsubsection{Bestimmung des Trägheitsmoments einer humanoiden Holzpuppe}
\label{subs:Bestimmung des Trägheitsmoments einer humanoiden Holzpuppe}

Analog zu 5.1.1 werden nun die experimentellen Trägheitsmomente der Puppe in ihren zwei Haltungen berechnet.
Die Messwerte sind:

\begin{table}
  \centering
  \caption{Periodendauern $T$ für Figur.}
  \begin{tabular}{c S[table-format=1,3] S[table-format=1,3]}
    \toprule
    $Messung$ & $T_{\text{Pos1}}\:/\:\si{\seconds}$ &$T_{\text{Pos2}}\:/\:\si{\seconds} $\\
    \midrule
    1&   0,568  &  0.814    \\
    2&   0,564  &  0.806    \\
    3&   0,59   &  0.814    \\
    4&   0,552  &  0.818    \\
    5&   0,572  &  0.814    \\
    6&   0,504  &  0.812    \\
    7&   0,552  &  0.806    \\
    \midrule
    Mittelwert  & 0,557  & 0,812\\
    Fehler      & 0,027  & 0,004\\
    \bottomrule
  \end{tabular}
\end{table}

und es folgen also
\begin{align*}
    I_{\text{Pos1}} &= \frac{D}{4\pi^2}T_{\text{Pos1}}^2
    = \begin{cases} 
    D_{\text{dyn}}  &
    \num{1,61(0,16)e-4}\si{\kilogram\metre\squared}\\
    D_{\text{stat}} &
    \num{2,4(0,5)e-4}\si{\kilogram\metre\squared}
  \end{cases}
    I_{\text{Pos2}} &= \frac{D}{4\pi^2}T_{\text{Pos2}}^2 
    = \begin{cases}
    D_{\text{dyn}} &
    \num{3,42(0,06)e-4}\si{\kilogram\metre\squared}\\
    D_{\text{stat}} &
    \num{5,1(1)e-4}\si{\kilogram\metre\squared}
    \end{cases}.
\end{align*}

Zur Berechnung der jeweiligen Trägheitsmomente 
sind alle Teile der Puppe an verschiedenen 
Stellen ausgemessen worden.
Über alle wird hier gemittelt und
es wird von einer homogenen Massenverteilung ausgegangen. 
Das heißt, die Masse eine Teils der Puppe
ist proportional zum Volumen de Teils 
im Vergleich zum Gesamtvolumen.
Alle Einzelteile der Puppe werden als Zylinder angenommen.

\begin{table}
  
  \caption{Länge und verschiedene Dicken der einzelnen Körperteile.}
  \begin{tabular}{c S[table-format=2,2] S[table-format=2,2] S[table-format=2,2] S[table-format=2,2] S[table-format=2,2]}
    \toprule
    $Teil  $ & $l\:/\:\si{\milli\metre}$ & $d\:/\:\si{\milli\metre} $ & $d\:/\:\si{\milli\metre} $ & $d\:/\:\si{\milli\metre} $ &
    $ d\:/\:\si{\milli\metre}$ & $d \:/\:\si{\milli\metre} $ \\
    \midrule
    \text{Figur_\text{ganz}}   & 252    \\
    \text{Kopf}                & 54,48   &  31,00   &   25,40   & 15,64     \\
    \text{Torso}               & 98,22   &  38,72   &   26,10   &  40,12    & 37,00 \\
    \text{Arm_\text{rechts}}   & 141,00  &  17,16   &   11,52   &   9,70    & 14,60 & 16,54 \\
    \text{Arm_\text{links}}    & 139,10  &  16,30   &   11,40   &   9,92    & 14,60 & 16,80 \\
    \text{Bein_\text{rechts}}  & 154,30  &  17,00   &   31,38   &  16,90    & 12,30 & 17,10 \\
    \text{Bein_\text{links}}   & 149,32  &  16,68   &   20,90   &  16,58    & 12,76 & 16,10 \\
    \bottomrule
  \end{tabular}
\end{table}

Es ergeben sich für die einzelnen Teile durch Mittelung der Werte für beide Arme und Beine:

\begin{table}
    \caption{Mittelwerte der Längen und Dicken aus Tabelle 5.}
    \begin{tabular}{c S[table-format=2,2] @{${}\pm{}$} S[table-format=2,2] S[table-format=2,2] S[table-format=2,2] @{${}\pm{}$} S[table-format=2,2]}
    \toprule 
    \text{Teil}\multicolumn {2}{c}{$l\:/\:\si{\milli\metre}$} & \multicolumn {2}{c}{$d\:/\:\si{\milli\metre}$}  \\
    \midrule

    \text{Kopf}   & 54,48     &     &  24,01 & 7,77   \\
    \text{Torso}  & 98,22     &     &  35,49 & 6,39   \\
    \text{Arm}    & 140,05 & 1,34   &  13,54 & 2,946   \\
    \text{Bein}   & 151,81 & 3,52   &  17,77 & 5,346   \\
    \bottomrule
    \end{tabular}
\end{table}

Damit ist das Gesamtvolumen der Figur
\begin{equation*}
    V_\text{ges} = \num{1,36(0,35)e-4} \si{\meter\cubed}
\end{equation*}

und die Massen der Glieder mit 
\begin{align*}
  m_x &= \frac{V_\text{x}}{V_\text{ges}}m_{\text{ges}}}\\
  \text{und}\\ 
  m_{\text{ges}} &= \num{162,53}\si{\gram}.
\end{align*}


\begin{table}
    \caption{Massenverteilung auf die einzelnen Körperteile.}
    \begin{tabular}{c S[table-format=2,1] @{${}\pm{}$} S[table-format=2,1] }
    \toprule
    Teil   & \multicolumn {2}{c}{$m\:/\:\si{\gram}$} //
    \midrule

    \text{Kopf}                &  29,5 &  17     \\
    \text{Torso}               &  64,3 &  19,6   \\
    \text{Arm}                 &  12   &  5,4    \\
    \text{Bein}                &  22,3 &  10,7     \\
    \bottomrule
    \end{tabular}
\end{table}

Mit diesen Werten dergibt sich für die einzelnen Trägheitsmomente
\begin{align*}
    I_{\text{Kopf}} &= \num{1,5(1,9)e-6}\si{\kilogram\metre\squared}\\
    I_{\text{Torso}} &= \num{7,1(4,8)e-6}\si{\kilogram\metre\squared}\\
    I_{\text{Arm}} &=   \num{1,9(1,6)e-7}\si{\kilogram\metre\squared}\\
    I_{\text{Bein}} &=  \num{2,5(0,8)e-6}\si{\kilogram\metre\squared}\\

\end{align*}

und mit dem Steinerschen Satz folgt dann mit dem Abstand $a$ von der Drehachse gleich dem Radius des Torsos
\begin{align*}
    I_{\text{ges,Pos1}} &= I_{\text{Kopf}}+I{\text{Torso}}+2\cdot(I_{\text{Arm}}+
    I_{\text{Bein}}+a^2(m_{\text{Arm}}+m_{\text{Bein}}))  &=\num{2,5(0,8)e-5}\si{\kilogram\metre\squared}.
\end{align*}

Da die einzige Änderung der Pose, von Pose1 auf Pose2 die ist, dass die Arme nach oben gestreckt sind anstatt am Körper anzuligen und sich daher der Abstand zur Drehachse nicht ändert ist auch das theoretische Trägheitsmoment das selbe.
