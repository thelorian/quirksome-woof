%!TEX program = lualatex
\documentclass{scrartcl}
\usepackage{amsmath}
\usepackage{amssymb}
\usepackage{siunitx}
\usepackage{unicode-math}
\usepackage{mathtools}
\usepackage{fontspec}
\usepackage{polyglossia}



\begin{document}

\section{Auswertung}
\label{sec:Auswertung}

\subsection{Bestimmung des Schubmoduls G}

Zur Berechnung des Schubmoduls wird () verwendet.
Die Größen $m_k$ und $R_k$ waren mit Fehler am Gerät abzulesen,
die restlichen Größen ergeben sich als Mittelwert der gemessenen Werte:

\begin{table}
\centering
\caption{Gemessener Durchmesser des Drahtes}
\label{tab:Dicke}
\begin{tabular}{ c S[table-format=1.3] }
\toprule
$\text{Messung#}$ & $2R/mm$ \\
\midrule
1 & 0,185   \\
2 & 0,190   \\
3 & 0,185   \\
4 & 0,188   \\
5 & 0,191   \\
\midrule
Mittelwert () & 0.188 \\
Fehler () & 0.001\\
\bottomrule
\end{tabular}
\end{table}

\begin{table}
\centering
\caption{Gemessener Durchmesser des Drahtes}
\label{tab:Dicke}
\begin{tabular}{ c S[table-format=2.1] }
\toprule
$\text{Messung#}$ & $L/cm$ \\
\midrule
1 & 60,0   \\
2 & 60.0   \\
3 & 60.1   \\
\midrule
Mittelwert () & 60.0 \\
Fehler () & 0.1 \\
\bottomrule
\end{tabular}
\end{table}

\begin{table}
\centering
\caption{Periodendauern ohne Modifikation}
\label{tab:Dauern}
\begin{tabular}{ c S[table-format=2.3] }
\toprule
$Messung#$ & $T/s$ \\
\midrule
1 & 18,065  \\
2 & 18,069  \\
3 & 18,055  \\
4 & 18,069  \\
5 & 18,053  \\
6 & 18,054  \\
7 & 18,063  \\
8 & 18,050  \\
9 & 18,057  \\
10 & 18,058 \\
\midrule
Mittelwert& 18,059 \\
Fehler () & 0,006\\
\bottomrule
\end{tabular}
\end{table}

Mit diesen Werten ergibt sich der Schubmodul als $G = (7,278 \pm 4,23\% )\cdot 10^{10} \frac{\text{N}}{m^2}$
mit einem prozentualen Fehler von .Zusammen mit dem zuvor angegebenen Elastizitätsmodul
ergibt sich dann mit () die Querkontraktionszahlt als $\mu = 0,4427$
mit dem selben realtiven Fehler da E als fehlerfrei angenommer wird und es sich um einen linearen Zusammenhang handelt.

\subsection{Messung des Erdmagnetfelds}

Das Magnetfeld von Helmholtz-Spulen ergibt sich ähnlich zu langen Spule durch ()
mit den angegebenen Werten $N = 390$ und $R_H = 78 mm$.
Die Periodendauern aus Tabelle(\ref{tab:DauernB}) werden werden im folgenden gegen $\frac{1}{T^2}$ aufgetragen:
\begin{figure}

\end{figure}
Die mit linearer Regression ausgerechnete Steigung der Geraden ist hier nach ()
dann genau das magnetische Moment $m= 2,51352 \cdot 10^{-3} Am^2$


\begin{table}
\centering
\caption{Periodendauern mit Helmholz-Spulen}
\label{tab:DauernB}
\begin{tabular}{ c S[table-format=1.1] S[table-format=1.1] S[table-format=2.3] }
\toprule
$Messung#$ & $I/A$ & $B/mT$ & $T/s$ \\
\midrule
1 & 1 & 4,5 & 14,044  \\
2 & 1 & 4,5 & 14,054  \\
3 & 1 & 4,5 & 14,034  \\
4 & 1 & 4,5 & 14,015  \\
5 & 1 & 4,5 & 14,017  \\
\midrule
Mittelwert ()  & & & 14,033 \\
Fehler ()      & & & 0,017        \\
\midrule
1 & 0,6 & 2,6 & 15,688  \\
2 & 0,6 & 2,6 & 15,673  \\
3 & 0,6 & 2,6 & 15,655  \\
4 & 0,6 & 2,6 & 15,675  \\
5 & 0,6 & 2,6 & 15,652  \\
\midrule
Mittelwert ()  & & & 15,668 \\
Fehler ()      & & &   0,015      \\
\midrule
1 & 0,8 & 1,8 & 16,738  \\
2 & 0,8 & 1,8 & 16,732  \\
3 & 0,8 & 1,8 & 16,722  \\
4 & 0,8 & 1,8 & 16,746  \\
5 & 0,8 & 1,8 & 16,714  \\
\midrule
Mittelwert ()  & & & 16,730 \\
Fehler ()       & & &    0,013                   \\
\midrule
1 & 0,8 & 1,3 & 17,354  \\
2 & 0,8 & 1,3 & 17,345  \\
3 & 0,8 & 1,3 & 17,331  \\
4 & 0,8 & 1,3 & 17,359  \\
5 & 0,8 & 1,3 & 17,344  \\
\midrule
Mittelwert ()  & & & 17,347 \\
Fehler ()      & & &   0,011                     \\
\midrule
1 & 0,8 & 0,9 & 18,092  \\
2 & 0,8 & 0,9 & 18,092  \\
3 & 0,8 & 0,9 & 18,064  \\
4 & 0,8 & 0,9 & 18,047  \\
5 & 0,8 & 0,9 & 18,041  \\
\midrule
Mittelwert () & & & 18,067\\
Fehler ()     & & &  0,024     \\
\bottomrule
\end{tabular}
\end{table}

Mit Hilfe des berechneten m kann nun das Erdmagnetfeld mit den gemessenen Werten bestimmt werden (\ref{tab:DauernE}).
Zuvor wird aber das gesamte Trägheitsmoment des Systems benötigt um () nach B umstellen zu können.
\begin{align*}
\Theta_{\text{ges}} = \Theta_{\text{Kugel}} + \Theta_k &= \frac{2}{5}m_k R_k^2 + \Theta_k = 1,34 \cdot 10^{-4} kgm^2\\
 \\ \text{Damit ist B dann:}\\
B^* &= \frac{4\pi^2 \Theta}{T_m^2m}-\frac{D}{m} =
\end{align*}


\begin{table}
\centering
\caption{Periodendauern mit Erdmagnetfeld}
\label{tab:DauernE}
\begin{tabular}{ c S[table-format=2.3] }
\toprule
$Messung#$ & $T/s$ \\
\midrule
1 & 19,687  \\
2 & 19,686  \\
3 & 19,691  \\
4 & 19,699  \\
5 & 19,700  \\
6 & 19,700  \\
7 & 19,712  \\
8 & 19,712  \\
9 & 19,713  \\
10 & 19,720 \\
\midrule
Mittelwert ()  & 19,702\\
Fehler ()     &  0,012\\
\bottomrule
\end{tabular}
\end{table}

\section{Diskussion}
\label{sec:Diskussion}

Abweichungen von den erwarteten Werten erwiesen sich ,bis auf das magnetische Moment, als wider erwarten gering,
da der Bereich der Kleinwinkelnäherung deutlich überschritten wurde und außerdem die Pendelbewegung der Kugel nicht vollständig unterbunden werden konnte.
Dank der hochempfindlichen und -präzisen Zeitmessung sind alle Ergebnisse außerdem bis auf systematische Fehler überdurchschnittlich genau.

\end{document}
