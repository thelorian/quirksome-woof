\section{Auswertung}
\label{sec:Auswertung}


\subsubsection{Niedrigdruck Messung}
\label{subs:Niedrigdruck Messung}

Zur Bestimmung der Verdampfungswärme aus den Messwerten wird die Lösung der Clausius-Clapeyronschen DGL ( ) genutzt.
\begin{equation*}
  \ln(p) = -\frac{L}{R}\cdot\frac{1}{T}
\end{equation*}
Die Steigung der Ausgleichsgeraden der aufgetragenen Werte für
$\ln(p)$ gegen $\frac{1}{T}$ mit linearer Regression ermöglicht
dann die Berechnung von $L$.
\begin{table}
\centering
\caption{Gemessener Durchmesser des Drahtes}
\label{tab:Dicke}
\begin{tabular}{ c S[table-format=1.3] }
\toprule
$ T/°C$ & $p/\si{\mili \bar}$ \\
\midrule
20 & 53    \\
24 & 68    \\
28 & 72    \\
32 & 79    \\
36 & 86    \\
40 & 96    \\
44 & 107   \\
48 & 121   \\
52 & 138   \\
56 & 165   \\
60 & 199   \\
64 & 238   \\
68 & 283   \\
72 & 336   \\
76 & 400   \\
80 & 475   \\
84 & 555   \\
88 & 649   \\
92 & 765   \\
96 & 875   \\
\midrule
Mittelwert () & 0.188 \\
Fehler () & 0.001\\
\bottomrule
\end{tabular}
\end{table}
