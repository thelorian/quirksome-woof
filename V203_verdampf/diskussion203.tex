\section{Diskussion}
\label{sec:Diskussion}

Zwar liegt der Literaturwert für die Verdampfungswärme von Wasser
($\SI{4,08e4}{\joule\per\mole}$) nicht innerhalb des Messfehlers des hier errechneten Wertes $\SI{4,220(12)e4}{\joule\per\mole}$, doch liegt dieser mit $3,4\%$ Abweichung im Rahmen der von dieser Messung zu erwartenden Genauigkeit.
Die wahrscheinlichste Fehlerquelle bei der Messung bestht in der manuell regelbarren Kühlung, die zum richtigen Zeitpunkt gedrosselt werden musste um das Ergebnis möglichst unverfälscht zu lassen,
aber auch die Abdichtungen der Apparatur waren nicht Fehlerfrei,
was man bemerken konnte da bei laufender evakuierender Wasserstrahlpumpe vor dem Beginn des Experiments der Druck bereits wieder um $\SI{25}{\mili\bar}$ gestiegen war.
Dass der errechnete Wert für $L_i$ nur marginal abweicht von $L_a$ zeig uns, 
dass der Großteil der Energie benötig wird um die Wasserstoffbrückenbindungen zu überwinden.